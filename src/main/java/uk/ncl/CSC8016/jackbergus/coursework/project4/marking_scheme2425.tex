\documentclass{article}

\usepackage[english]{babel}
\usepackage[utf8]{inputenc}
\usepackage{amsmath,amssymb,hyperref}
\usepackage{parskip}
\usepackage{graphicx}

\usepackage{newpxtext,newpxmath}
\usepackage[tagged, highstructure]{accessibility}
% Margins
\usepackage[top=2.5cm, left=3cm, right=3cm, bottom=4.0cm]{geometry}
% Colour table cells
\usepackage[table]{xcolor}

% Get larger line spacing in table
\newcommand{\tablespace}{\\[1.25mm]}
\newcommand\Tstrut{\rule{0pt}{2.6ex}}         % = `top' strut
\newcommand\tstrut{\rule{0pt}{2.0ex}}         % = `top' strut
\newcommand\Bstrut{\rule[-0.9ex]{0pt}{0pt}}   % = `bottom' strut

%%%%%%%%%%%%%%%%%
%     Title     %
%%%%%%%%%%%%%%%%%
\title{Coursework CSC8016}
\author{Giacomo Bergami}
%\date{12\textsuperscript{th} of February, 2024}

\begin{document}
	\maketitle
	
	\section*{Use Case Scenario}
We want to implement a bank system, whether the threads are
either clients logging to the servers, or the bank server containing
the information for each account. Each client interacts with the server through the \texttt{BankFacade} by accessing an \texttt{openTransaction} method, through which the each client can create multiple accesses to its bank account. This mimicks the possibility of multitasking operations via mobile phone, ATM, web banking. Each client can only \textsf{pay} or \textsf{withdraw} money, and gets the list of the total movements after \textsf{commit}-ting the operation. In a realistic scenarios, transactions might also \textsf{abort}; e.g.:
\begin{itemize}
\item The ATM receives the request of money, but the server fails and does
not deduct the money from the on-line account.
\item The server fails after deducting the from the on-line account, but no
money is given at the ATM
\end{itemize}
Still, all of the operations done by the thread over the specific bank account will be confirmed and will become effective only when the transaction is going to be \textbf{commit}ted. After a commit or an abort, no further operation through that transaction is allowed, thus requiring the user to open another transaction within the same thread. More details on the operations' requirements are given in the final marking scheme.


As in any industrial setting where teams split up the duties,
you are assigned an API that you need to implement. Such an API is provided both on Canvas and at \url{https://github.com/jackbergus/NCL_CSC8016/tree/main/src/main/java/uk/ncl/CSC8016/jackbergus/coursework/project4}. This will then require to extend the \texttt{BankFacade} for implementing the definition of novel transactions, and the implementation of \texttt{TransactionCommands} for performing the client-server communication. The student is free to choose whichever is the best way to make these two entities communicate. E.g., the bank could be either modelled as a finer-grained monitor, but inside this monitor at least one thread per logged user should be running; also, such a bank could be also implemented as a consumer threads handling all of the clients' messages. 

	
	\section*{Assumptions}
	\begin{itemize}
		\item In a realistic example, communications happen between processes via UDP messages. In this module, we don't require that. We can freely assume that each client (ATM, OnLine Banking) is mimicked by one single
		thread. We assume they directly exploit such an interface (no
		FrontEnd is required!)
		\item  If the bank is implemented as a server, such a thread might receive the ``client messages'' through shared
		variables. 
		\item The server should also keep track of the transactions
		that are performed for handling commit/abort correctly. You are not required to tolerate the server crash (this is more of
		a back-up task rather than a concurrent programming one), but
		you must tolerate the client ones (that is more related to
		concurrent programs’ management)!
		\item We assume that the \texttt{BankFacade} class is initialized with the users having an account in their bank as well as the balance associated to that (constructor with \texttt{HashMap<String, Double>}). The system should not allow to open/close new bank accounts.
		\item The server should allow multiple users logging in running contemporarily on distinct bank account. In order to maximise seriality and concurrency requirements, the students might investigate \textit{optimistic protocols} for transactions, but this is not strictly required.
		\item The actual implementation provided both in Canvas and on GitHub provides incorrect solutions for the following reasons:
		\begin{enumerate}
		\item The execution of the process is \textbf{wrongly} \underline{always} sequential, as
		one client always blocks the other clients from
		accessing the bank!
		\item Client's crashes are \textbf{wrongly} \underline{not} tolerated, as those are always committing the operations to then
		main log without checking whether the action was
		effectively performed or not!
		\item Somehow, the computations are ``logically'' correct, that is \textsf{pay}, \textsf{withdraw}, \textsf{commit}, and \textsf{abort} implement the expected semantics. Still, this is not sufficient for passing the coursework with full marks. 
		\end{enumerate}
	\end{itemize}
	
	\section*{Submission Requirements}
	\begin{enumerate}
\item To help us with the marking, the students should update the static method \texttt{StudentID}  in \texttt{BankFacade} so to return a string corresponding to their own student id. This will help us expediting the marking using our automated tool.
		\item \texttt{Solution} should be the only class to be changed, as the current implementation does not pass the provided tests! 

		\item Submit the code as a zipped \textit{Maven} project via \texttt{File > Export > Project to Zip file\dots} with \textbf{no} \textit{jar} and \textit{classes}. The source code will be recompiled from scratch, and no pre-compiled jar/class is going to be run.
		 
		\item If you want to use an external Java library, please consider the following:
		\begin{itemize}
			\item The Java library should be explicitly described as a \texttt{<dependency>} in the \texttt{pom.xml} file, and should only access the libraries from the default \textit{Maven Central Repository}.
			\item A library might provide single concurrency mechanisms primitives, but not ready-made solutions already composing those: semaphores, monitors, locks, just logs, thread barriers, thread pools, passing le baton mechanisms are allowed. Code reuse from the exercises and examples seen in class is permitted.
			
			\item Systems completely solving the coursework for you are \textbf{strictly prohibited}: e.g., any kind of (data) management system having concurrency control (ensuring safe concurrent thread access to any data representation) and supporting concurrent transactions (implementing any kind of transaction protocol, either pessimistic or optimistic) \textbf{must be avoided}, as they both implement commit/aborts and thread-safe operations on data. 
			\item None of the (direct or indirect) dependencies of the coursework should rely on external servers or processes to run or to be installed.
			\item The solution should \textbf{not} include external jar files.
			\item If unsure whether the solution might be exploited, please ask before submitting.
		\end{itemize}
		 
		\item Attached to the source code, please provide a short report motivating the compliance of the source code to each point and sub-point of the marking scheme. Providing such report in form of comments in the implementation is also fine. New classes might be created for supporting the implementation, but  existing classes should be neither renamed or moved to a different package.
		
	\end{enumerate}


	\section*{Marking Scheme} 
\renewcommand{\labelitemii}{$\blacksquare$}
	The marking scheme is capped at  \textbf{100\%}.
	\begin{itemize}
		\item Single-Thread Correctness \textbf{[+50\%]}
			\begin{description}
			\item [+10\%:] I cannot open a transaction if the user does not appear in the initialization map.
			\item [+10\%:] I can always open a transaction if the user, on the other hand, appears on the initialization map.
	\begin{itemize}
				\item When no further operations are performed (pay/withdraw), the returned \texttt{totalAmount} reflects the amount on the Bank.
			\end{itemize}
			\item [+15\%:] After committing a transaction, the results provides the total changes into the account.
			\begin{itemize}
				\item The returned commit information should contain relevant \textit{Operation} of \textit{OperationType} on the bank account (pay/withdrawal) before the commit.
				\item After paying money into the account, the final total amount is the sum of the previous amount of money and the amount being paid, as retrieved through the \texttt{CommitResult}.
				\item I cannot re-commit or abort a closed transaction (I shall return \texttt{null} rather than a \texttt{CommitResult}).
				
			\end{itemize}

			\item [+15\%: ] No overdraft is allowed.
			\begin{itemize}
				\item I can always withdraw $0.0$ money from my account.
				\item I can never withdraw an amount of money which is greater than the amount at my disposal.
				\item The commit should list all of the operations, thus including the attempt to overdraft.
				\item The operation causing the overdraft shall not be considered as an \textit{unsuccessful} operation, rather than one ignored within the transaction.
			\end{itemize}

%			\item [+4\%:] I cannot interact with a blog thread if this was not previously created..
%\begin{itemize}
%\item You should not be able to remove a non-existing topic thread.
%\item You cannot post a comment over a non-existing topic thread.
%\item You cannot read the posted messages for a non-existing topic thread.
%\end{itemize}
%			\item [+16\%:] I can always interact with a topic thread that was previously created.
%\begin{itemize}
%\item You can create a new topic thread.
%\item You can post a message within an existing  topic thread.
%\item You can correctly read all the messages in their order of appearance.
%\end{itemize}
%
%\item [+4\%:] I am correctly handling the thread closure for an existing topic thread.
%
%\item [+7\%:] I am correctly handling the \texttt{pollForUpdate} method where, if successful requests are always fired before polling for events, should always return the most recent event available.
%\begin{itemize}
%\item You only retrieve the latest update event.
%\item You can correctly detect all the required interaction events of the users over the platform.
%\end{itemize}
%\item [+3\%:] If no user performs any suitable action, \texttt{pollForUpdate} should pause indefinitely.
%
%\item[+10\%: ] I am correctly handling the \texttt{set} method from \texttt{ReadWriteMonitorMultiRead} (please see the documentation of this method for additional information concerning its expected behaviour).
%\item[+6\%: ] I am correctly handling the \texttt{get} method from \texttt{ReadWriteMonitorMultiRead} (please see the documentation of this method for additional information concerning its expected behaviour).




		\end{description}


	\item  Multi-Threaded Correctness \textbf{[+50\%]}
\begin{description}
	\item [+10\%:] A single user can open concurrent transactions.
	\begin{itemize}
		\item As other concurrent transactions are not committed yet, each thread can only see the committed statues from their account.
		\item Any user should be allowed to concurrently log in in its account (e.g., through different possible devices).
		\item Under the assumption that user's threads never abort, the final amount of the money in the bank account correspond to the overall total of pays and withdraws.
	\end{itemize}
	\item [+10\%:] Multiple users can open concurrent transactions. \textit{[Same requirements as above, plus the following:]}
\begin{itemize}
	\item Different users should never run serially, as they operate on different accounts (The code unrealistically assumes that neither  pay other accounts nor receive money from them).
\end{itemize}
\item [+10\%:] No dirty reads allowed.
\begin{itemize}
	\item A thread shall never see in the total account balance after the commit the amount of money being paid/withdrawn before other threads' commits.
	\item A thread shall never be able to withdraw money paid by other non-committing threads. 
	\item Ignored operations should consider the unsuccessful withdraws from the account.
\end{itemize} 
\item [+10\%:] Handing aborted transactions:
\begin{itemize}
	\item Aborted transactions apply no change to the bank account.
	\item In particular, aborted threads should leave the bank in a consistent state.
\end{itemize}
%	\item [+7\%:] The concurrent creation of different topics is handled correctly.
%\begin{itemize}
%\item Creating multiple topic threads does not raise deadlocks.
%\item All the topics requests being sent are fulfilled and available in the exact same order of creation.
%\end{itemize}
%	\item [+9\%:] The concurrent posting of multiple messages within multiple topic threads is handled correctly.
%\begin{itemize}
%\item Posting multiple messages across topics or within the same topics does not result into deadlocks.
%\item All the messages being sent are fulfilled and available in the exact order of posting.
%\end{itemize}
%\item [+12\%:] We have one querying user, one moderator user, and eight answering user. Each of those are simulated as distinct threads. The \textit{querying thread} posts the topic and publishes the first comment within the thread; after receiving exactly 8 answers from the answering threads, the querying user posts a thank you message. The \textit{answering threads} are waiting for the first message from the querying user to be posted, after which they provide a reply within the thread. The \textit{moderator thread}  will close all the topics reaching a number of 10 comments.   
%
%\item [+12\%:] While having only one single user running and one subscriber to receive the updates from the website, no interference occurs, and all the perceived events actually match the expected results.
\end{description}


\item  Advanced Features (capped at maximum 10\%):
\begin{itemize}
	\item \textbf{[+1\%]} The program allows to visually determine the correctness of the operations performed by the threads (e.g., terminal prints, engaging with extra tests not provided to the students, or graphical user interfaces).
	\item \textbf{[+1\%]} Any Java library imported via \texttt{pom.xml} `not violating the 3\textsuperscript{rd} Submission Requirement. The following libraries will not be considered, as already provided in the given source code as dependencies: \texttt{annotations} from \texttt{org.jetbrains}, and \texttt{jansi} from \texttt{org.fusesource\-.jansi}.
	\item  \textbf{[+1\%]} Either the methods to be implemented or the extended tests exploit Java's concurrent collections.
\item \textbf{[+1\%]} The student correctly uses \texttt{ReentrantReadWriteLocks} within the \texttt{Solution} class.
\item \textbf{[+1\%]} The student re-implemented \texttt{ReentrantReadWriteLocks} using the solution discussed in the advanced material on the slides (self-study).
\item \textbf{[+3\%]} The solution provides a satisfactory implementation of the optimistic protocol by not actually updating the hash map at each operation, but by enabling the operation as per current Bank state (as seen by a fresh read operation). Journal solutions are used to replay and test the operations, to double check whether they are still valid.
	\item \textbf{[+1\%]} Thread pools are used to either handle multiple requests from multiple users, or start multiple threads within the extended tests.
	\item  \textbf{[+5\%]} The Bank service is emulated realistically as a separate thread accepting restful requests; this requires the Blog to handle requests one at a time through a queue; Still, the student shall not change the API interface as currently provided.






	%\item \textbf{[+2\%]} The student correctly exploits semaphores.
	%\item \textbf{[+2\%]} The student exploited the optimistic transaction principle, where multiple users can log-in (not only the same user multiple times!).
	%\item \textbf{[+2\%]} Usage of monitors or multithreaded producers and consumers on the interaction with \texttt{Blog} (semaphores might be also exploited).


\end{itemize}
	\end{itemize}


	
\end{document}

